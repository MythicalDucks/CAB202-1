%!TEX TS-program = xelatex
%!TEX options = -aux-directory=Debug -shell-escape -file-line-error -interaction=nonstopmode -halt-on-error -synctex=1 "%DOC%"
\documentclass{article}
% Packages

%% Math enhancements
\usepackage{amsmath} % Misc enhancements to math equations
\usepackage{cancel} % Draw diagonal lines and arrows in math equations
\usepackage{mathtools} % Starred versions of amsmath matrix environments; Multiline, cases, gathered environment
\usepackage{chngcntr} % Reset counter within sections
\usepackage{interval} % Format intervals
\intervalconfig{
    soft open fences
}

%% Symbols
\usepackage{amssymb} % Extended symbol collection - also loads amsfonts
\usepackage{stmaryrd} % Extra symbols

%% Fonts
\usepackage{mathrsfs} % Support \mathcal and \mathscr

%% Environments
\usepackage{amsthm} % Use theorems

%% Tables and arrays
\usepackage{booktabs} % Top and bottom rule for tabular
\usepackage{tabularx} % Advanced Tables

%% Lists
\usepackage{enumitem} % Itemize, enumerate, description environments

%% Page layout
\usepackage{geometry} % Page layout customisation
\usepackage{fancyhdr} % Page headers and footers
\usepackage{float} % Float objects such as figures and tables
\usepackage{tcolorbox} % Create boxed environments

%% Text enhancements
\usepackage[none]{hyphenat} % Disable hyphenation of long text
\usepackage{ragged2e} % Text alignment options

%% Referencing
\usepackage{tocbibind} % Adds bibliography to the Table of Contents
\usepackage{url} % Define urls

%% Graphics
\usepackage{graphicx} % Extension to graphics
% \graphicspath{ {./figures/} }

%% Miscellaneous
\usepackage[outputdir=Debug]{minted} % Typeset programming code
\usepackage{siunitx} % SI units package
\usepackage{derivative} % Derivative notation
\usepackage{pdfpages} % Import PDFs into document

\usepackage[hidelinks]{hyperref} % Handle cross-referencing
\usepackage{bookmark} % New bookmark organisation for hyperref

%% Unicode setup
\usepackage[warnings-off={mathtools-colon, mathtools-overbracket}]{unicode-math}
\setmathfont{Latin Modern Math}
\setmathfont[range={bb, bbit}, Scale=MatchUppercase]{TeX Gyre Pagella Math}
\setmathfont[range={\mathcal, \mathbfcal}, StylisticSet=1]{XITS Math}
\setmathfont[range={\mathscr}]{XITS Math}
\setmathfont[range={"2205}]{XITS Math} % chktex 18

% Preamble

%% Misc Commands

%%% Number Sets
\newcommand*{\N}{\mathbb{N}}
\newcommand*{\Z}{\mathbb{Z}}
\newcommand*{\Q}{\mathbb{Q}}
\newcommand*{\I}{\mathbb{I}}
\newcommand*{\R}{\mathbb{R}}
\newcommand*{\C}{\mathbb{C}}

%%% Empty set character
\let\oldemptyset\emptyset
\let\varnothing\relax
\newcommand{\varnothing}{\char"2205} % chktex 18

%%% Contradiction
\newcommand{\contradiction}{
    \hspace{-1em}
	{\hbox{
	\setbox0=\hbox{\(\mkern-3mu{\times}\mkern-3mu\)}
	\setbox1=\hbox to0pt{\hss\copy0\hss}
	\copy0\raisebox{0.5\wd0}{\copy1}\raisebox{-0.5\wd0}{\box1}\box0}}
}

%%% Lines for matrices
\newcommand*{\vertbar}{\rule[-1ex]{0.5pt}{2.5ex}}
\newcommand*{\horzbar}{\rule[.5ex]{2.5ex}{0.5pt}}

%% Paired Delimiters
\DeclarePairedDelimiter{\ceil}{\lceil}{\rceil}
\DeclarePairedDelimiter{\floor}{\lfloor}{\rfloor}
\DeclarePairedDelimiter{\abracket}{\langle}{\rangle}
\DeclarePairedDelimiter{\abs}{\lvert}{\rvert}
\DeclarePairedDelimiter{\norm}{\lVert}{\rVert}

%% Probability Functions
\let\Pr\relax
\DeclareMathOperator{\Pr}{Pr}
\DeclareMathOperator{\E}{E}
\DeclareMathOperator{\Var}{Var}
\DeclareMathOperator{\Cov}{Cov}
\DeclareMathOperator{\Corr}{Corr}

\newcommand{\Perm}[2]{\prescript{#1}{}{P}_{#2}}

%% Hyperbolic Functions
\DeclareMathOperator{\arcsinh}{arcsinh}
\DeclareMathOperator{\arccosh}{arccosh}
\DeclareMathOperator{\arctanh}{arctanh}
\DeclareMathOperator{\arccoth}{arccoth}
\DeclareMathOperator{\arcsech}{arcsech}
\DeclareMathOperator{\arccsch}{arccsch}

%% Linear Algebra
%%% Augmented matrices
\makeatletter
\renewcommand*\env@matrix[1][*\c@MaxMatrixCols c]{%
  \hskip -\arraycolsep
  \let\@ifnextchar\new@ifnextchar
  \array{#1}}
\makeatother

%%% Operators
\let\det\relax
\DeclareMathOperator{\det}{det}
\DeclareMathOperator{\Tr}{Tr}
\DeclareMathOperator{\diag}{diag}
\DeclareMathOperator{\adj}{adj}

\DeclareMathOperator{\vspan}{span}
\DeclareMathOperator{\vref}{ref}
\DeclareMathOperator{\vrref}{rref}

\DeclareMathOperator{\vrank}{rank}
\DeclareMathOperator{\vnull}{null}

\DeclareMathOperator{\proj}{proj}

\DeclareMathOperator{\vim}{im}
\DeclareMathOperator{\vcoim}{coim}
\DeclareMathOperator{\vker}{ker}
\DeclareMathOperator{\vcoker}{coker}

\newcommand{\columnspace}[1]{\mathcal{C}\left(\symbf{#1}\right)}
\newcommand{\rowspace}[1]{\mathcal{C}\left(\symbf{#1}^{\top}\right)}
\newcommand{\nullspace}[1]{\mathcal{N}\left(\symbf{#1}\right)}
\newcommand{\leftnullspace}[1]{\mathcal{N}\left(\symbf{#1}^{\top}\right)}

%% Additional operators
\DeclareMathOperator{\erf}{erf}

% Theorems
\theoremstyle{definition}
\newtheorem{definition}{Definition}[section]

\theoremstyle{plain}
\newtheorem{theorem}{Theorem}[subsection]
\newtheorem{corollary}{Corollary}[theorem]
\newtheorem{lemma}{Lemma}[theorem]
\newtheorem{axiom}{Axiom}

\theoremstyle{remark}
\newtheorem{remark}{Remark}
\newtheorem{note}{Note}[subsection]
\newtheorem*{statement}{Statement}

\newenvironment{examples}[1][Examples]{\let\qed\relax\proof[#1]\mbox{}\\*}{\endproof}
\newenvironment{question}[1][Question]{\let\qed\relax\proof[#1]\mbox{}\\*}{\endproof}
\newenvironment{solution}[1][Solution]{\let\qed\relax\proof[#1]\mbox{}\\*}{\endproof}

\newenvironment{proofcase}[1]{\proof[Case #1]\mbox{}}{\endproof}

%% Box styles
\tcbuselibrary{skins}
\newtcolorbox{tcolorboxlarge}[1][]{
    skin=enhanced,
    boxrule=0pt,
    frame hidden,
    sharp corners,
    borderline west={0.5pt}{0pt}{black},
    borderline east={0.5pt}{0pt}{black},
    enlarge left by=10pt,
    width=\linewidth-20pt,
    opacityback=0,
    coltitle=black,
    fonttitle=\large\bfseries,
    #1
}

\newtcolorbox{tcolorboxcols}[1][]{
    skin=enhanced,
    boxrule=0pt,
    frame hidden,
    sharp corners,
    borderline west={0.5pt}{0pt}{black},
    opacityback=0,
    coltitle=black,
    fonttitle=\large\bfseries,
    #1
}

%% Reset counter within subsections
\counterwithin*{equation}{section}
\counterwithin*{equation}{subsection}
\counterwithin*{remark}{subsection}

%% Page layout setup
\pagestyle{fancy}
\setlength\headheight{24pt}
\setlength\parindent{0pt} % Indent first line of new paragraphs


% Additional packages & macros

% Header and footer
\newcommand{\unitName}{Microprocessors and Digital Systems}
\newcommand{\unitTime}{Semester 2, 2022}
\newcommand{\unitCoordinator}{Dr Mark Broadmeadow}
\newcommand{\documentAuthors}{Tarang Janawalkar}

\fancyhead[L]{\unitName}
\fancyhead[R]{\leftmark}
\fancyfoot[C]{\thepage}

% Copyright
\usepackage[
    type={CC},
    modifier={by-nc-sa},
    version={4.0},
    imagewidth={5em},
    hyphenation={raggedright}
]{doclicense}

\date{}

\begin{document}
%
\begin{titlepage}
    \vspace*{\fill}
    \begin{center}
        \LARGE{\textbf{\unitName}} \\[0.1in]
        \normalsize{\unitTime} \\[0.2in]
        \normalsize\textit{\unitCoordinator} \\[0.2in]
        \documentAuthors
    \end{center}
    \vspace*{\fill}
    \doclicenseThis
    \thispagestyle{empty}
\end{titlepage}
\newpage
%
\tableofcontents
\newpage
%
\section{Microcontroller Fundamentals}
\subsection{Computer}
\begin{definition}[Computer]
    A computer is a digital electronic machine that can be programmed to carry
    out sequences of arithmetic or logical operations (computation) automatically.
\end{definition}
\begin{definition}[Control unit]
    The control unit interprets the instructions and decides what actions to take.
\end{definition}
\begin{definition}[Arithmetic logic unit]
    The arithmetic logic unit (ALU) performs computations required by the control unit.
    This store data in memory, stores program instructions in memory

\end{definition}

microprocessor doesnt include external storage, inputs, outputs,

mircocontroller, integrated circuit, single chip that contains a CPU, memory, peripherals and is a standalone system.

on a desktop motherboard, the CPU is the microprocessor and the memory, and other peripherals are connected via other busses on the motherboard.

The QUTy uses a microcontroller called ATtiny1626, that are within a family of microcontrollers called AVRs.

\subsection{ATtiny1626 Microcontroller}
The ATtiny1626 microcontroller has the following features:
\begin{itemize}
    \item CPU:\@ AVR Core (AVRxt variant)
    \item Memory:\@
          \begin{itemize}
              \item Flash memory (16KB) used to store program instructions in memory
              \item SRAM used (2KB) to store data in memory
              \item EEPROM (256B)
          \end{itemize}
    \item Peripherals, i.e., inputs/outputs. These are implemented in hardware (part of the chip) in order to offload complexity
\end{itemize}
\subsubsection{Flash Memory}
\begin{itemize}
    \item Non-volatile --- memory is not lost when power is removed
    \item Inexpensive
    \item Slower than SRAM
    \item Can only erase in large chunks
    \item Typically used to store programme data
    \item Generally read-only. Programmed via an external tool, which is loaded once and remains static during the lifetime of the program
    \item Writing is slow
\end{itemize}
\subsubsection{SRAM}
\begin{itemize}
    \item Volatile --- memory is lost when power is removed
    \item Expensive
    \item Faster than flash memory and is used to store variables and temporary data
    \item Can access individual bytes (large chunk erases are not required)
\end{itemize}
\subsubsection{EEPROM}
\begin{itemize}
    \item Older technology
    \item Expensive
    \item Non-volatile
    \item Can erase individual bytes
\end{itemize}
\subsection{AVR Core}
\begin{definition}[Computer programme]
    A computer programme is a sequence or set of instructions in a programming language
    for a computer to execute.
\end{definition}
The main function of the AVR Core Central Processing Unit (CPU) is to ensure correct program execution.
The CPU must, therefore, be able to access memories, perform calculations, control peripherals, and handle interrupts.
Some key characteristics of the AVR Core are:
\begin{itemize}
    \item 8-bit Reduced Instruction Set Computer (RISC)
    \item 32 working registers (r0 to r31)
    \item Program Counter (PC) --- location in memory where the program is stored
    \item Status Register --- modified by the ALU and stores key information from the ALU (i.e., whether a result is negative)
    \item Stack Pointer --- temporary data that doesn't fit into the registers
    \item 8-bit core --- all data, registers, and operations, operate within 8-bits
\end{itemize}
\subsection{Programme Execution}
At the time of reset PC = 0.
\begin{enumerate}
    \item Fetch instruction (from memory)
    \item Decode instruction (decode binary instruction)
    \item Execute instruction:
          \begin{itemize}
              \item Execute an operation
              \item Store data in data memory, the ALU, a register, or update the stack pointer
          \end{itemize}
    \item Store result
    \item Update PC (move to next instruction or if instruction is longer than 1, add 2. Or move to different location in program arbitrary address k)
\end{enumerate}
\subsection{Instructions}
\begin{itemize}
    \item The CPU understands and can execute a limited set of instructions --- \textasciitilde88 unique instructions for the ATtiny1626
    \item Instructions are encoded in programme memory as opcodes. Most instructions are two bytes long, but some instructions are four bytes long
    \item The AVR Instruction Set Manual describes all of the available instructions, and how they are translated into opcodes
    \item Instructions fall loosely into five categories:
          \begin{itemize}
              \item Arithmetic and logic --- ALU
              \item Change of flow --- jumping to different sections of the code or making decisions
              \item Data transfer --- moving data in/out of registers, into the data space, or into RAM
              \item Bit and bit-test --- looking at data in registers (which bits are set or not set)
              \item Control --- changing what the CPU is doing
          \end{itemize}
\end{itemize}

ldi - load immediate
takes constat value and stores it in a register

data must be in a register to be used in an instruction

can only access r16-r31

\subsection{Interacting with memory and peripherals}
\begin{itemize}
    \item The CPU interacts with both memory and peripherals via the data space
    \item From the perspective of the CPU, the data space is single large array of
          locations that can be read from, or written to, using an address
    \item We control peripherals by reading from, and writing to, their registers
    \item Each peripheral register is assigned a unique address in the data space
    \item When a peripheral is accessed in this manner we refer to it as being
          memory mapped, as we access them as if they were normal memory
    \item Different devices, peripherals and memory can be included in a memory map
          (and sometimes a device can be accessed at multiple different addresses)
\end{itemize}
\subsection{Memory map}

\subsection{Assembly code}
\begin{itemize}
    \item The opcodes that get placed into programme memory are what we call
          machine code (i.e. code the machine operates on directly)
    \item We tend not to write machine code directly as it is:
    \item Not human readable
    \item Prone to errors (swapping a single bit can completely change the operation)
    \item Instead we can write instructions directly in assembly code
    \item We use instruction mnemonics to specify each instruction
    \item You have already seen some of these: ldi, add, sts, jmp …
    \item An assembler takes the assembly code and translates it into opcodes that can
          be loaded into programme memory
\end{itemize}
\end{document}
